\documentclass[11pt]{article}
\usepackage{hyperref}
\usepackage{wrapfig}
\usepackage{alltt}
\usepackage{amsmath}
\usepackage{minibox}
\usepackage{graphicx}
\oddsidemargin -0.0in
\topmargin -0.5in
\textwidth 6.25in
\textheight 9.25in
\setlength\parindent{0pt}
\setlength\parskip{10pt}

\newcommand{\ttt}[1]{\texttt{#1}}

\begin{document}

\begin{center}
  \Large CSCI 141 - Fall 2019 \\
  \Large Lab 3: Conditionals and Boolean Logic\\
  \Large Due Date: Friday, October 18th at 9:59pm
\end{center}

\section*{Introduction}
This lab gives you practice with \ttt{if} statments (also sometimes called
\textit{conditional}, or \textit{selection} statements. In the process, you'll
also get some more experience with Boolean operators. The idea is the following:
your goal is to write a program that recommends what clothing items to wear
based on the weather conditions indicated by a user.

If you have questions, be sure to ask the TA: your TA is there to help you!  By
now you've seen how to use Thonny on both your Windows and Linux accounts. You
are free to select whichever operating system you want to use.

\section{Setup}
We recommend creating a new directory/folder called \textit{lab3} on your N
drive (Windows) or in your home directory (Linux). In your lab3 directory,
create a new Python file \texttt{clothing\_picker.py}.

\section{Unary Selection}
\textit{Unary selection} is a fancy name for a simple \texttt{if} statement.
You've seen these already in lecture: the \texttt{if} statement allows you to
execute a sequence of statements (or a \textit{code block}) if a given
\textbf{boolean expression} evaluates to \ttt{True}, or skip over the code block
if the expression evaluates to \ttt{False}. 

The code block inside an \ttt{if} statement must contain one or more statements.
In Python, the code block associated with an \ttt{if} statement is distinguished
by indenting lines of code immediately underneath the line containing the
\texttt{if} keyword of the selection statement. The syntax and structure of a
unary selection statement are shown below:

\begin{alltt}
if \textit{boolean_expression}:
    \textit{statement\_1}
    \textit{statement\_2}
    \textit{statement\_3}
\end{alltt}

For this first version, write a single unary selection statement that checks
whether the user has specified whether it is windy or not.  If it is windy, the
program should tell the user not to bring an umbrella.
Pseudocode and sample input and output for this first version of your program
are given below. 

\begin{itemize}
\setlength\itemsep{0em}
\item Ask the user if it is windy
\item Save user input into a variable
\item If it is windy, print ``Don't bring an umbrella because it's windy.''
\item If it is not windy, do nothing
\end{itemize}

\begin{figure}
  \centering
  \includegraphics[height=1.25in]{ifwindy.png}
  \caption{Sample output for the initial version.}
\end{figure}

\section{Binary Selection}

We've also discussed binary selection, which is a fancy name for an
\ttt{if}/\ttt{else} statement. It has an \ttt{if} clause and an indented code
block just as in unary selection, but it also has an \ttt{else} clause and code
block that is executed whenever the Boolean expression in the if clause
evaluates to \ttt{False}.  The syntax and structure of a binary selection
statement are shown below below, where statements 1 through 3 are the code block
for the \ttt{if} clause, and statements 4 and 5 constitute the code block for
the \ttt{else} clause.

\begin{alltt}
if \textit{boolean_expression}:
    \textit{statement\_1}
    \textit{statement\_2}
    \textit{statement\_3}
else:
    \textit{statement\_4}
    \textit{statement\_5}
\end{alltt}

Next, modify your code so that it still prompts the user to answer whether it is
windy, but this time if the answer is ``no'', then have the program output, ``It is
not windy. Bring an umbrella if you wish.'' If it is windy, the program should
output the same as before. Sample input and output for this second version of
your program is shown below.

\begin{figure}[h]
\centering
    \includegraphics[height=0.75in]{ifelsewindy.png}
  \caption{Sample output, Binary Selection}
\end{figure}

\pagebreak
\section{Boolean expression with logical operators}

We've discussed in lecture how to use more complicated boolean expressions;
specifically, the logical operators \texttt{or}, \texttt{and}, and \texttt{not}
were presented.  Modify your code to also prompt the user to for whether it is
sunny or cloudy. Retain the \texttt{if, else} code as you've already written,
except change the Boolean expression to check if it is windy and sunny. If the
user specifies yes, then the output should be ``It's windy and sunny, so bundle
up and don't bring an umbrella.'' If it is not both windy and sunny, have the
program output, ``It is not both windy and sunny.''

Pseudocode for the revised version of your program is shown below. Sample output
is shown in Figure \ref{fig:windsun}.
\begin{itemize}
\setlength\itemsep{0em}
\item Ask user if it is windy, save input  into a variable
\item Ask user if it is sunny or cloudy, save input into a second variable
\item If it is windy and sunny, output ``It's windy and sunny, so bundle up and
  don't bring an umbrella.''
\item Otherwise, output ``It is not both windy and sunny.''
\end{itemize}

\textbf{Note:} Here and in all further parts of the lab, you may assume that the
user responds to the sunny/cloudy prompt with the exact input \ttt{"sunny"} or
\ttt{"cloudy"}; you do not need to handle other inputs. Also notice that the
instructions above are phrased only in terms of \texttt{sunny} and \texttt{not
sunny}.

\begin{figure}[h]
\centering
    \includegraphics[height=1.25in]{windsun.png}
  \caption{Sample output for boolean expression with logical
  operators\label{fig:windsun}}
\end{figure}

\newpage
\section{Nested if statements}

As shown in lecture, it is possible to nest an entire selection statement
(\ttt{if} statement with an \ttt{else} clause) inside of a code block of an
existing \ttt{if} statement. The syntax is shown in below. To make it
easier to see, a box has been drawn around the outer-most and inner most if
statements.
\begin{alltt}
if \textit{boolean_expression_1}:
    \minibox[frame]{if \textit{boolean_expression_2}:\\
    \textit{statement_1}\\
else:\\
    \textit{statement_2}}
else:
    \textit{statement\_4}
    \textit{statement\_5}
\end{alltt}

Modify your code so that the outer condition
(\ttt{\textit{boolean\_expression1}}) checks if it is windy, and the inner
condition (\ttt{\textit{boolean\_expression2}}) checks whether it is sunny. If it is
windy and sunny, print ``It is windy and sunny,''; if it is windy and not sunny,
print ``It is windy and not sunny''; if it is not windy, print  ``It is not
windy.'' Sample output is shown In Figure~\ref{nestedIf}.

\begin{figure}[h]
\centering
    \includegraphics[height=1.7in]{nested.png}
\caption{Output of program with nested conditionals}
\label{nestedIf}
\end{figure}

\section{Clothing picker: chained conditionals}

One more possible clause in a conditional statement an \ttt{elif} clause. An
elif, which is short of else if, contains a Boolean expression that is checked
ONLY if the first \ttt{if}'s condition, and all preceding
\ttt{elif} conditions all evaluate to False. Unlike an \ttt{else}, whose code
block is ALWAYS executed if the if condition evaluates to False, the code block
of an elif is executed only if the conditional of the elif evaluates to True.
Note that including an \ttt{else} is never syntactically required; an \ttt{if}
statement can have zero or more \ttt{elif} clauses and zero or one \ttt{else}
clause. In the program you submit, you should make sure you're demonstrating at
least one use of an \ttt{else} clause.

This time, have your program prompt the user to ask if it is sunny, and also ask
for the \textbf{temperature}.  Modify your program so that the \ttt{else} code
block has a nested if statement as well. Both of the nested if statements will
now have \ttt{if}, \ttt{elif}, and \ttt{else} clauses (see sample below). The
outer condition should check whether it is sunny, and inner \ttt{if/elif/else}
should make recommendations according to the temperature and whether or not it
is sunny. Don't forget to convert the temperature input to an \ttt{int} and
recall that you can use comparison operators like \ttt{<=} and \ttt{>=} to get
the boolean result of numerical comparisons.

Write appropriate conditions that rely on the user's input (whether it is sunny and the temperature), and write appropriate print statements that produce the clothing recommendations indicated in Figure~\ref{pickerGoal}.

\begin{figure}[h]
\centering
    \includegraphics[width=0.8\linewidth]{Picture2.png}
\caption{Schematic of logic for clothing picker}
  \label{pickerGoal}
\end{figure}

Table~\ref{testTable} shows the sunny/temperature combinations and their
corresponding output value (clothing recommendation) that your program should
print. Sample output is shown in Figure~\ref{fig:pickerOutput}.

\begin{center}
\begin{table}
\caption{Sample input/output combinations}
    \label{testTable}
  \begin{tabular}{ | l | l | p{9cm}|}
    \hline
    Sunny & Temperature & Output \\
    \hline
    Yes &	Less than 60 degrees &	Wear a sweater \\
    \hline
    Yes	& 60 degrees exactly	 & Woo hoo, it is 60 degrees. Wear what you want \\
        \hline
    Yes	& More than 60 degrees &	Wear a t-shirt and flip flops \\
        \hline
    No	& Less than 40 degrees	& Wear a coat and hat \\
        \hline
    No	& Between 40 and 50 degrees	& Not quite freezing, but close. Bundle up \\
        \hline
    No &	50 degrees exactly	& A jacket is best \\
        \hline
    No &	More than 50 degrees	& Wear a long sleeved shirt \\
  \hline
  \end{tabular}
  \end{table}
\end{center}

\begin{figure}[!h]
\centering
    \includegraphics[height=4in]{chained.png}
    \caption{Sample clothing picker program output. Note that this does not
    display all possible cases in the table above, but your program must work
    for all of them.}
  \label{fig:pickerOutput}
\end{figure}

\section*{Submission}

Upload \texttt{clothing\_picker.py} to Canvas for grading.

\section*{Rubric}

\begin{center}
  \begin{tabular}{ |p{13cm}|l|}
    \hline
    Your file is called \texttt{clothing\_picker.py} & 1 point \\
    \hline
    The top of \ttt{clothing\_picker.py} has comments including
    your name, date, and a short description of the program's purpose. Comments
    placed throughout the code explain what the code is doing. & 3 \\
  \hline
    Your program makes use of \ttt{if}, \ttt{elif}, and \ttt{else}. & 6 \\
    \hline
    Your program correctly prompts the user and stores the user's input. & 3 \\
    \hline
    Your code provides unique clothing combinations for each of the sunny/temperature combinations in the table in this lab handout. & 7 \\
    \hline
    Total & 20 points\\
    \hline
  \end{tabular}
\end{center}

\end{document}
